\documentclass[12pt]{article}

\usepackage[utf8]{inputenc}
\usepackage[margin=2.5cm]{geometry}
\usepackage{graphicx}
\usepackage{lipsum} % solo para texto de ejemplo
\usepackage{titling}
\usepackage{multicol}
\usepackage{ragged2e}

\begin{document}

%---------------- PORTADA -------------------
\vspace*{2cm}

\begin{center}
    \Huge\textbf{IA en Acción: Manual Práctico para Dominar la Inteligencia Artificial}\\[1.5cm]
    \large Dra. Marlene Méndez Moreno\\[5cm]
\end{center}


\vfill


\begin{flushright}
\begin{tabular}{rl}
\textbf{Integrantes:} & \\[0.3cm]
Crismar Chan Chin & 7612 \\
Roberto May Vergara & 7630 \\
Adiel Uc Peralta & 766 \\
Berenise K\'u Centeno & 7628\\
Pedro Chi Ek & 7614 \\
\end{tabular}
\end{flushright}

\vspace{1cm}

\begin{center}
Ingeniería en Sistemas Computacionales\\
Materia: Inteligencia Artificial\\
Octavo  Semestre\\
Fecha: 10 de Abril del 2025
\end{center}

\newpage

%---------------- ÍNDICE -------------------

\begin{center}
    \Huge \textbf{Índice}
\end{center}
\vspace{1cm}

\begin{itemize}
%---------------- investigacion 1 : Representación del conocimiento, razonamiento -----------------
    \item Principios y Metodología de la Inteligencia Artificial
    \begin{itemize}
    
    
   		\item Concepto inicial e Historia,
		\item Como fueron establecidos los Principios y/o Paradigmas
		\item Por quién fue creado - A quién se le ocurrió
		\item Curiosidades del tema
		\item Elementos que conforman cada tema.
		\item 2 ejemplos aplicativos de cada tema y Empresas de desarrollo lo usan
    
     
    \end{itemize}
    
    \item Paradigmas de la Inteligencia Artificial.
    \begin{itemize}
		\item Concepto inicial e Historia,
		\item Como fueron establecidos los Principios y/o Paradigmas
		\item Por quién fue creado - A quién se le ocurrió
		\item Curiosidades del tema
		\item Elementos que conforman cada tema.
		\item 2 ejemplos aplicativos de cada tema y Empresas de desarrollo lo usan
    \end{itemize}
    %---------------- investigacion 2 : Aspectos Metodológicos en Inteligencia Artificial ----------
    
    \item Mapas conceptuales.
		    \begin{itemize}
\item Concepto inicial e Historia,
\item Como fueron establecidos los Principios y/o Paradigmas
\item Por quién fue creado - A quién se le ocurrió
\item Curiosidades del tema
\item Elementos que conforman cada tema.
\item Donde inicio el uso de estos aspectos metodológicos
	\item En donde se usa actualmente.
\item 2 ejemplos aplicativos de cada tema y
Empresas de desarrollo lo usan
    \end{itemize}    
    
\item Redes semánticas.
    \begin{itemize}
\item Concepto inicial e Historia,
\item Como fueron establecidos los Principios y/o Paradigmas
\item Por quién fue creado - A quién se le ocurrió
\item Curiosidades del tema
\item Elementos que conforman cada tema.
\item Donde inicio el uso de estos aspectos metodológicos
	\item En donde se usa actualmente.
\item 2 ejemplos aplicativos de cada tema y
Empresas de desarrollo lo usan
    \end{itemize}

\item Razonamiento monótono.
    \begin{itemize}
\item Concepto inicial e Historia,
\item Como fueron establecidos los Principios y/o Paradigmas
\item Por quién fue creado - A quién se le ocurrió
\item Curiosidades del tema
\item Elementos que conforman cada tema.
\item Donde inicio el uso de estos aspectos metodológicos
	\item En donde se usa actualmente.
\item 2 ejemplos aplicativos de cada tema y
Empresas de desarrollo lo usan
    \end{itemize}

\item Conocimiento no-monótono y otras lógicas.
    \begin{itemize}
\item Concepto inicial e Historia,
\item Como fueron establecidos los Principios y/o Paradigmas
\item Por quién fue creado - A quién se le ocurrió
\item Curiosidades del tema
\item Elementos que conforman cada tema.
\item Donde inicio el uso de estos aspectos metodológicos
	\item En donde se usa actualmente.
\item 2 ejemplos aplicativos de cada tema y
Empresas de desarrollo lo usan
    \end{itemize}

\item Razonamiento probabilístico.
    \begin{itemize}
\item Concepto inicial e Historia,
\item Como fueron establecidos los Principios y/o Paradigmas
\item Por quién fue creado - A quién se le ocurrió
\item Curiosidades del tema
\item Elementos que conforman cada tema.
\item Donde inicio el uso de estos aspectos metodológicos
	\item En donde se usa actualmente.
\item 2 ejemplos aplicativos de cada tema y
Empresas de desarrollo lo usan
    \end{itemize}

\item Teorema de Bayes.
    \begin{itemize}
\item Concepto inicial e Historia,
\item Como fueron establecidos los Principios y/o Paradigmas
\item Por quién fue creado - A quién se le ocurrió
\item Curiosidades del tema
\item Elementos que conforman cada tema.
\item Donde inicio el uso de estos aspectos metodológicos
	\item En donde se usa actualmente.
\item 2 ejemplos aplicativos de cada tema y
Empresas de desarrollo lo usan
    \end{itemize}
    
    
   %---------------- investigacion 3 : Reglas de busqueda--------------   
    
    
    
	\item Representación de conocimiento mediante reglas
	\begin{itemize}
	   \item Concepto inicial e Historia,
	\item Como fueron establecidos los Principios y/o Paradigmas
	\item Por quién fue creado - A quién se le ocurrió
	\item Curiosidades del tema
	\item Elementos que conforman cada tema.
	\item Donde inicio el uso de estos aspectos metodológicos
	\item En donde se usa actualmente.
	\item 2 ejemplos aplicativos de cada tema y Empresas de desarrollo lo usan
	\end{itemize}
	
	\item Métodos de Inferencia en reglas
			\begin{itemize}
	   \item Concepto inicial e Historia,
	\item Como fueron establecidos los Principios y/o Paradigmas
	\item Por quién fue creado - A quién se le ocurrió
	\item Curiosidades del tema
	\item Elementos que conforman cada tema.
	\item Donde inicio el uso de estos aspectos metodológicos
	\item En donde se usa actualmente.
	\item 2 ejemplos aplicativos de cada tema y Empresas de desarrollo lo usan
	\end{itemize}
	
	\item Reglas de producción.
		\begin{itemize}
	   \item Concepto inicial e Historia,
	\item Como fueron establecidos los Principios y/o Paradigmas
	\item Por quién fue creado - A quién se le ocurrió
	\item Curiosidades del tema
	\item Elementos que conforman cada tema.
	\item Donde inicio el uso de estos aspectos metodológicos
	\item En donde se usa actualmente.
	\item 2 ejemplos aplicativos de cada tema y Empresas de desarrollo lo usan
	\end{itemize}
	
	\item Sintaxis de las reglas de producción.
			\begin{itemize}
	   \item Concepto inicial e Historia,
	\item Como fueron establecidos los Principios y/o Paradigmas
	\item Por quién fue creado - A quién se le ocurrió
	\item Curiosidades del tema
	\item Elementos que conforman cada tema.
	\item Donde inicio el uso de estos aspectos metodológicos
	\item En donde se usa actualmente.
	\item 2 ejemplos aplicativos de cada tema y Empresas de desarrollo lo usan
	\end{itemize}
	
	\item Semántica de las reglas de producción.
			\begin{itemize}
	   \item Concepto inicial e Historia,
	\item Como fueron establecidos los Principios y/o Paradigmas
	\item Por quién fue creado - A quién se le ocurrió
	\item Curiosidades del tema
	\item Elementos que conforman cada tema.
	\item Donde inicio el uso de estos aspectos metodológicos
	\item En donde se usa actualmente.
	\item 2 ejemplos aplicativos de cada tema y Empresas de desarrollo lo usan
	\end{itemize}
	
	\item Arquitectura de un sistema de Producción (SP) o sistemas basados en reglas, (SBR).
		\begin{itemize}
        \item Hechos
        \item Base de conocimientos.
        \item Mecanismo de control
           \item Concepto inicial e Historia,
	\item Como fueron establecidos los Principios y/o Paradigmas
	\item Por quién fue creado - A quién se le ocurrió
	\item Curiosidades del tema
	\item Elementos que conforman cada tema.
	\item Donde inicio el uso de estos aspectos metodológicos
	\item En donde se usa actualmente.
	\item 2 ejemplos aplicativos de cada tema y Empresas de desarrollo lo usan
   	 \end{itemize}
	
	\item Espacios de estados determinísticos y espacios no determinísticos.
			\begin{itemize}
	   \item Concepto inicial e Historia,
	\item Como fueron establecidos los Principios y/o Paradigmas
	\item Por quién fue creado - A quién se le ocurrió
	\item Curiosidades del tema
	\item Elementos que conforman cada tema.
	\item Donde inicio el uso de estos aspectos metodológicos
	\item En donde se usa actualmente.
	\item 2 ejemplos aplicativos de cada tema y Empresas de desarrollo lo usan
	\end{itemize}
	
	\item Búsqueda sistemática
		\begin{itemize}
		\item Búsqueda de metas en profundidad
		\item Búsqueda de metas en  anchura
		   \item Concepto inicial e Historia,
	\item Como fueron establecidos los Principios y/o Paradigmas
	\item Por quién fue creado - A quién se le ocurrió
	\item Curiosidades del tema
	\item Elementos que conforman cada tema.
	\item Donde inicio el uso de estos aspectos metodológicos
	\item En donde se usa actualmente.
	\item 2 ejemplos aplicativos de cada tema y Empresas de desarrollo lo usan
		\end{itemize}
	

    
     %---------------- investigacion 4 : Aplicaciones Tecnicas de IA -------------------   
    
    
    \item  Robótica
		\begin{itemize}
			\item Conceptos básicos
			\item Clasificación
			\item Desarrollos actuales y aplicaciones
			\item Concepto inicial e Historia,
			\item Como fueron establecidos los Principios y/o Paradigmas
			\item Por quién fue creado - A quién se le ocurrió
			\item Curiosidades del tema
			\item Elementos que conforman cada tema.
			\item Donde inicio el uso de estos aspectos metodológicos
			\item En donde se usa actualmente.
			 \item 2 ejemplos aplicativos de cada tema y Empresas de desarrollo lo usan
		\end{itemize}
    
    
    \item  Redes Neuronales (RN)
    		\begin{itemize}
			\item Conceptos básicos
			\item Clasificación
			\item Desarrollos actuales y aplicaciones
			\item Concepto inicial e Historia,
			\item Como fueron establecidos los Principios y/o Paradigmas
			\item Por quién fue creado - A quién se le ocurrió
			\item Curiosidades del tema
			\item Elementos que conforman cada tema.
			\item Donde inicio el uso de estos aspectos metodológicos
			\item En donde se usa actualmente.
			 \item 2 ejemplos aplicativos de cada tema y Empresas de desarrollo lo usan
		\end{itemize}
    
    \item Visión artificial
    		\begin{itemize}
			\item Conceptos básicos
			\item Desarrollos actuales y aplicaciones
			\item Concepto inicial e Historia,
			\item Como fueron establecidos los Principios y/o Paradigmas
			\item Por quién fue creado - A quién se le ocurrió
			\item Curiosidades del tema
			\item Elementos que conforman cada tema.
			\item Donde inicio el uso de estos aspectos metodológicos
			\item En donde se usa actualmente.
			 \item 2 ejemplos aplicativos de cada tema y Empresas de desarrollo lo usan
		\end{itemize}
    
    \item Lógica difusa (Fuzzy Logic)
    		\begin{itemize}
			\item Conceptos básicos
			\item Desarrollos actuales y aplicaciones
			\item Concepto inicial e Historia,
			\item Como fueron establecidos los Principios y/o Paradigmas
			\item Por quién fue creado - A quién se le ocurrió
			\item Curiosidades del tema
			\item Elementos que conforman cada tema.
			\item Donde inicio el uso de estos aspectos metodológicos
			\item En donde se usa actualmente.
			 \item 2 ejemplos aplicativos de cada tema y Empresas de desarrollo lo usan
		\end{itemize}
    
    \item Procesamiento de Lenguaje Natural (PLN)
    		\begin{itemize}
			\item Conceptos básicos
			\item Desarrollos actuales y aplicaciones
			\item Concepto inicial e Historia,
			\item Como fueron establecidos los Principios y/o Paradigmas
			\item Por quién fue creado - A quién se le ocurrió
			\item Curiosidades del tema
			\item Elementos que conforman cada tema.
			\item Donde inicio el uso de estos aspectos metodológicos
			\item En donde se usa actualmente.
			 \item 2 ejemplos aplicativos de cada tema y Empresas de desarrollo lo usan
		\end{itemize}
    
    \item Sistemas Expertos (SE)
    		\begin{itemize}
			\item Conceptos básicos
			\item Clasificación
			\item Desarrollos actuales y aplicaciones
			\item Concepto inicial e Historia,
			\item Como fueron establecidos los Principios y/o Paradigmas
			\item Por quién fue creado - A quién se le ocurrió
			\item Curiosidades del tema
			\item Elementos que conforman cada tema.
			\item Donde inicio el uso de estos aspectos metodológicos
			\item En donde se usa actualmente.
			 \item 2 ejemplos aplicativos de cada tema y Empresas de desarrollo lo usan
		\end{itemize}
    
    
    
    
  
    
\end{itemize}



\end{document}